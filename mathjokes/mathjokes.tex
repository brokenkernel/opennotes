\documentclass{article}
\usepackage{cancel}
\usepackage{amsmath}% http://ctan.org/pkg/amsmath

\title {"Mathematical Jokes"}
\author {"Eitan Adler"}

\begin {document}
\maketitle

\section{Cancellation}
\begin{equation}
	\begin{split}
		\frac{64}{16}
		=
		\frac{\cancel{6}4}{1\cancel{6}}
		=
		\frac{4}{1}
	\end{split}
\end{equation}
\begin{equation}
	\begin{split}
		\frac{95}{19}
		=
		\frac{\cancel{9}5}{1\cancel{9}}
		=
		\frac{5}{1}
		=
		5
	\end{split}
\end{equation}
\begin{equation}
	\begin{split}
		\frac{65}{26}
		=
		\frac{\cancel{6}5}{2\cancel{6}}
		=
		\frac{5}{2}
	\end{split}
\end{equation}
\begin{equation}
	\begin{split}
		\frac{98}{49}
		=
		\frac{\cancel{9}8}{4\cancel{9}}
		=
		\frac{8}{4}
		=
		2
	\end{split}
\end{equation}

\section{1 = 2}

\begin{equation}
	\begin{split}
		b        &= a \\
		ab       &= a^2 \\
		ab - b^2 &= a^2 - b^2 \\
		b(a - b) &= (a + b)(a - b) \\
		b\cancel{(a - b)} &= (a + b)\cancel{a - b)} \\
		b        &= a + b \\
		b        &= b + b \\
		b        &= 2b \\
		1        &= 2
	\end{split}
\end{equation}



\end {document}
