\documentclass{article}
\usepackage{cancel}
\usepackage{amsmath}% http://ctan.org/pkg/amsmath

\title{'Mathematical Jokes'}
\author{Eitan Adler'}

\begin{document}
\maketitle


\section{Cancellation}
\subsubsection{Arithmetic}

\begin{equation}
	\begin{split}
		\frac{64}{16}
		=
		\frac{\cancel{6}4}{1\cancel{6}}
		=
		\frac{4}{1}
	\end{split}
\end{equation}

\begin{equation}
	\begin{split}
		\frac{95}{19}
		=
		\frac{\cancel{9}5}{1\cancel{9}}
		=
		\frac{5}{1}
		=
		5
	\end{split}
\end{equation}

\begin{equation}
	\begin{split}
		\frac{65}{26}
		=
		\frac{\cancel{6}5}{2\cancel{6}}
		=
		\frac{5}{2}
	\end{split}
\end{equation}

\begin{equation}
	\begin{split}
		\frac{98}{49}
		=
		\frac{\cancel{9}8}{4\cancel{9}}
		=
		\frac{8}{4}
		=
		2
	\end{split}
\end{equation}

\begin{equation}
	\begin{split}
		\frac{163}{326}
		=
		\frac{1\cancel{6}\cancel{3}}{\cancel{3}2\cancel{6}}
		=
		\frac{1}{2}
	\end{split}
\end{equation}

\begin{equation}
	\begin{split}
		\frac{2666}{6665}
		=
		\frac{2\cancel{6}66}{\cancel{6}665}
		=
		\frac{2\cancel{6}6}{\cancel{6}65}
		=
		\frac{2\cancel{6}}{\cancel{6}5}
		=
	    \frac{2}{5} =
	\end{split}
\end{equation}



\subsubsection{Calculus}
\begin{equation}
	\begin{split}
		\frac{d}{dx}
		\frac{1}{x}
		=
		\frac{\cancel{d}}{\cancel{d}x}
		\frac{1}{x}
		=
		\frac{}{x}
		\frac{1}{x}
		=
		- \frac{1}{x^2}
	\end{split}
\end{equation}

\begin{equation}
	\begin{split}
		\frac{d}{dx} 0 =
		\frac{\cancel{d}}{\cancel{d}x} 0
		=
		-\frac{0}{x}
		=
		0
	\end{split}
\end{equation}


\subsection{roots}

\begin{equation}
	\begin{split}
	     \sqrt[6]{64} = \sqrt[\cancel{6}]{\cancel{6}4} = \sqrt{4} = 2
	\end{split}
\end{equation}

\section{1 = 2}

\begin{equation}
	\begin{split}
		b        &= a \\
		ab       &= a^2 \\
		ab - b^2 &= a^2 - b^2 \\
		b(a - b) &= (a + b)(a - b) \\
		b\cancel{(a - b)} &= (a + b)\cancel{(a - b)} \\
		b        &= a + b \\
		b        &= b + b \\
		b        &= 2b \\
		1        &= 2
	\end{split}
\end{equation}

\section{limits}

\begin{equation}
	\begin{split}
		(\lim_{x\to8^+}\frac{1}{x-8} = \infty) \\
		\implies \\
		(\lim_{x\to3^+}\frac{1}{x-8} = \omega) \\
		(\lim_{x\to3^-}\frac{1}{x-8} = m)
	\end{split}
\end{equation}


\end{document}