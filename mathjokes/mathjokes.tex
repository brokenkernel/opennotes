\documentclass{article}
\usepackage[utf8]{inputenc}
\usepackage[english]{babel}
\usepackage{cancel}
\usepackage{amsmath, amssymb}% http://ctan.org/pkg/amsmath
\usepackage{csquotes}
\usepackage[backend=biber]{biblatex}
\addbibresource{cites.bib}


\title{'Mathematical Jokes'}
\date{}
\author{}


\newcommand{\iu}{\mathrm{i}} % roman "i"
\newcommand{\justif}[2]{&{#1}&\text{#2}}

\begin{document}
\maketitle
\begin{abstract}
	A collection of math jokes.
\end{abstract}

\tableofcontents
\clearpage

\section{Cancellation}
\subsection{Arithmetic}

\begin{equation}
	\begin{split}
		\frac{64}{16}
		=
		\frac{\cancel{6}4}{1\cancel{6}}
		=
		\frac{4}{1}
	\end{split}
\end{equation}

\begin{equation}
	\begin{split}
		\frac{95}{19}
		=
		\frac{\cancel{9}5}{1\cancel{9}}
		=
		\frac{5}{1}
		=
		5
	\end{split}
\end{equation}

\begin{equation}
	\begin{split}
		\frac{65}{26}
		=
		\frac{\cancel{6}5}{2\cancel{6}}
		=
		\frac{5}{2}
	\end{split}
\end{equation}

\begin{equation}
	\begin{split}
		\frac{98}{49}
		=
		\frac{\cancel{9}8}{4\cancel{9}}
		=
		\frac{8}{4}
		=
		2
	\end{split}
\end{equation}

\begin{equation}
	\begin{split}
		\frac{163}{326}
		=
		\frac{1\cancel{6}\cancel{3}}{\cancel{3}2\cancel{6}}
		=
		\frac{1}{2}
	\end{split}
\end{equation}

\subsubsection{chains}

\begin{equation}
	\begin{split}
		\frac{2666}{6665}
		=
		\frac{2\cancel{6}66}{\cancel{6}665}
		=
		\frac{2\cancel{6}6}{\cancel{6}65}
		=
		\frac{2\cancel{6}}{\cancel{6}5}
		=
	    \frac{2}{5} =
	\end{split}
\end{equation}



\subsection{Calculus}
\begin{equation}
	\begin{split}
		\frac{d}{dx}
		\frac{1}{x}
		=
		\frac{\cancel{d}}{\cancel{d}x}
		\frac{1}{x}
		=
		\frac{}{x}
		\frac{1}{x}
		=
		- \frac{1}{x^2}
	\end{split}
\end{equation}

\begin{equation}
	\begin{split}
		\frac{d}{dx} 0 =
		\frac{\cancel{d}}{\cancel{d}x} 0
		=
		-\frac{0}{x}
		=
		0
	\end{split}
\end{equation}


\subsection{Roots}

\begin{equation}
	\begin{split}
	     \sqrt[6]{64} = \sqrt[\cancel{6}]{\cancel{6}4} = \sqrt{4} = 2
	\end{split}
\end{equation}

\section{1 = 2}

\subsection{Cancellation}

%\begin{equation}
%%	\begin{split}
	\begin{subequations}
	\begin{align}
		b        &= a  \justif{\quad}{by assumption} \label{subeqn:cancel-1} \\
		ab       &= a^2 \justif{\quad}{by \ref{subeqn:cancel-1}} \label{subeqn:cancel-2} \\
		ab - b^2 &= a^2 - b^2 \justif{\quad}{by \ref{subeqn:cancel-1}} \label{subeqn:cancel-3} \\
		b(a - b) &= (a + b)(a - b) \justif{\quad}{by \ref{subeqn:cancel-1}}  \label{subeqn:cancel-4} \\
		b\cancel{(a - b)} &= (a + b)\cancel{(a - b)} \justif{\quad}{by cancelation \cite{defn_of_cancel}} \label{subeqn:cancel-5} \\
		b        &= a + b \justif{\quad}{by \ref{subeqn:cancel-1}} \label{subeqn:cancel-6} \\
		b        &= b + b \justif{\quad}{by \ref{subeqn:cancel-1}} \label{subeqn:cancel-7} \\
		b        &= 2b \justif{\quad}{by division} \label{subeqn:cancel-8} \\
		1        &= 2 \justif{\quad}{by division} \label{subeqn:cancel-9}
	\end{align}
	\label{eqn:all-lines}
\end{subequations}
%%	\end{split}
%\end{equation}

\subsection{Repetition}

\begin{equation}
	\begin{split}
		x^2& = \underbrace{x+x+\cdots+x}_{(x\text{ times})} \\
		\frac{d}{dx}x^2& = \frac{d}{dx}[\underbrace{x+x+\cdots+x}_{(x\text{ times})}] \\
		2x& = \underbrace{1+1+\cdots+1}_{(x\text{ times})} \\
		2x& = x \\
		& \implies \\
		2& = 1
	\end{split}
\end{equation}

\subsection{Imaginary Numbers}

\begin{equation}
	\begin{split}
		1 = \sqrt{1} = \sqrt{(-1)\cdot(-1)} = \sqrt{-1}\cdot \sqrt{-1} = \iu \cdot \iu = -1
	\end{split}
\end{equation}

\section{Limits}

\begin{equation}
	\begin{split}
		(\lim_{x\to8^+}\frac{1}{x-8} = \infty) \\
		\implies \\
		(\lim_{x\to3^+}\frac{1}{x-8} = \omega) \\
		(\lim_{x\to3^-}\frac{1}{x-8} = m)
	\end{split}
\end{equation}

\nocite{*}
\printbibliography[heading=bibintoc]

\end{document}
